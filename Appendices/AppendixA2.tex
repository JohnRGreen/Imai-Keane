% Appendix section A.2
% LaTeX path to the root directory of the current project, from the directory in which this file resides
% and path to econtexPaths which defines the rest of the paths like \FigDir
\providecommand{\econtexRoot}{}\renewcommand{\econtexRoot}{..}
\providecommand{\econtexPaths}{}\renewcommand{\econtexPaths}{\econtexRoot/TeXtools/econtexPaths}
% Mimicing professor's "ugly" solution to enable sharing base code between local LaTeX compilation and Overleaf

\providecommand{\FigDir}{\econtexRoot/FigDir}
\providecommand{\DataDir}{\econtexRoot/Data}
\providecommand{\SectionDir}{\econtexRoot/Sections}
\providecommand{\AppDir}{\econtexRoot/Appendices}
\providecommand{\TablesDir}{\econtexRoot/Tables}

\documentclass[\econtexRoot/ImaiKeane]{subfiles}
%\onlyinsubfile{\externaldocument{\econtexRoot/ImaiKeane}}

\begin{document}
\quad A.2.     \textit{Simulated Likelihood Calculation.}     In this section, we describe the simulation steps that are required to construct the simulated likelihood function. \par
\medskip
\noindent \textit{Step 1.}      Simulate $\{K_t^m, h_t^m, C_t^m, A_t^m\}_{t=t_0}^T$ starting from the initial period as follows:
\begin{enumerate}
\item  Draw the true initial human capital $K_{t_0}^m$. \par
  First, draw the initial period measurement error $\xi_0$ and then, derive
  $$K_{t_0}^m = \dfrac{K_{t_0}^D}{\xi_0}$$
  \item  Draw the true initial asset $A_{t_0}^m$. \par
    If the initial asset is observed, then draw the measurement error $\xi_{3,t_0}$, and derive
    $$A_{t_0}^m =A_{t_0}^D - \xi_{3,t_0} $$
    If the initial asset is not observable, then draw $A_{t_0}^m$ from $N(\bar{A}, V_{\bar{A}})$.
  \item Draw the taste shock
    $$ \ln(\epsilon_2) \sim N(\frac{1}{2}\sigma_2^2,\sigma_2) $$
      and solve for the optimal consumption and labor supply. That is,
      \begin{equation*}
        \begin{split}
          \{C_{t_0}^m,h_{t_0}^m\} = \text{arg max}_{ \{C_{t_0},h_{t_0}\} }& \{u(C_{t_0},t_0) - v(h_{t_0}, \epsilon_{2,t_0}) + \beta E_{t_0} V_{t_0+1}[(1+r)A_{t_0} \\
          & + K_{t_0}^m h_{t_0} - C_{t_0}, \hat{K}_{t_0 + 1}, \epsilon_{t_0 + 1}]\}
        \end{split}
        \end{equation*}
        subject to
        \begin{equation*}
        \begin{split}
          A_{t_0+1} & = (1+r)A_{t_0}^m + K_{t_0}^m h_{t_0} - C_{t_0} \\
          \hat{K}_{t_0+1}& = g(h_{t_0},  K_{t_0}^m, t_0)
        \end{split}
      \end{equation*}
      Notice that we already have the polynomial approximation of the emax function
      $$V^E ( A_{t_0+1},\hat{K}_{t_0 + 1}) = E_{t_0}V_{t_0 + 1}(A_{t_0+1},\hat{K}_{t_0 + 1}, \epsilon_{1,t_0+1}, \epsilon_{2,t_0+1})$$
      from the DP solution, which we will use in this case.
    \item Draw the human capital shock $\epsilon_{1,t_0+1},$ and derive the next period state variables.\par
      That is,
       \begin{equation*}
        \begin{split}
          A_{t_0+1}^m & = (1+r)A_{t_0}^m + K_{t_0}^m h_{t_0}^m - C_{t_0}^m \\
          \hat{K}_{t_0+1}& = g(h_{t_0}^m,  K_{t_0}^m, t_0) \\
           K_{t_0+1} & =  \hat{K}_{t_0} \epsilon_{1,t_0+1}
        \end{split}
      \end{equation*}
    \item Now, repeat (3) and (4) until the end period $T$ to derive the sequence of variables $\{K_t^m,h_t^m,C_t^m,A_t^m\}_{t=t_0}^T$.
    \end{enumerate}
    \textit{Step 2.}     Given the simulated sequence of variables $\{K_t^m,h_t^m,C_t^m,A_t^m\}_{t=t_0}^T$, we then derive the measurement error. Then, we calculate the log likelihood increment for person $i$  at the $m$\textsuperscript{th} simulation draw as follows.\par
    Let us denote
           \begin{equation*}
        \begin{split}
          \xi_0^m &= \log K_{t_0}^D - \log K_{t_0}^m \\
          \xi_{1,t}^m &= \log K_t^D - \log h_t^D - \log K_t^m - \log h_t^m \\
          \xi_{2,t}^m &= h_t^D - h_t^m \\
           \xi_{3,t}^m &=  A_t^D - A_t^m \\
        \end{split}
      \end{equation*}
If the initial wage at period to is available in the data, then we construct the initial wage measurement error $\tilde{\xi}_0^m$ and the log density of $\tilde{\xi}_0^m$ becomes part of the log likelihood increment of person $i$. If wage and hours data for person $i$ at period $t > t_0$ is available, then we derive the labor income measurement error  $\tilde{\xi}_1^m$, and the log density of  $\tilde{\xi}_1^m$, becomes part of the log likelihood increment for person $i$ at period $t$. On the other hand, if either wage or hours xor both for person $i$ at period $t > t_0$ are not available, then the log likelihood increment for person $i$ at period $t$ does not contain any wage data $K_t^D$. Similarly, if hours data for person $i$ at period $t$ is available, then we derive the hours measurement error $\tilde{\xi}_2^m$, and the log density of $\tilde{\xi}_2^m$ becomes part of the log likelihood increment for person $i$ at period $t$. On the other hand, if hours data are not available for person $i$ at period $t$, then the log likelihood increment for person $i$ at period $t$ does not contain any hours information. Construction of the log likelihood increment for assets for person $i$ at period $t$ is done similarly. \par
Then, the log likelihood increment for person $i$ is
           \begin{equation*}
             \begin{split}
               l_i^m = & \sum_{t=t_0+1}^T \left[ \dfrac{(\tilde{\xi}_{1,t}^m+\frac{1}{2}\sigma_1)^2}{-2\sigma_1^2} - \log \sigma_1 - (\log K_t^D + \log h_t^d) \right] I(K_t^D, h_t^D \text{ observable }) \\
                       & + \sum_{t=t_0}^T \left[  \dfrac{(\tilde{\xi}_{2,t}^m)^2}{-2\sigma_1^2} - \log \sigma_1 \right] I(h_t^D \text{ observable }) \\
                       & + \sum_{t=t_0+1}^T \left[  \dfrac{(\tilde{\xi}_{3,t}^m)^2}{-2\sigma_3^2} - \log \sigma_3 \right] I(A_t^D \text{ observable }) \\
                       & + \left[ \dfrac{(A_{t_0}^m - \bar{A})^2}{-2 \sigma_{\bar{A}}^2} - \log \sigma_{\bar{A}} \right] I(A_t^D \text{ observable }) \\
               & + \left[ \dfrac{ ( \tilde{\xi}_0^m + \frac{1}{2} \sigma_0)^2}{-2 \sigma_0^2} - \log \sigma_0 - \log K_{t_0}^D \right] I(K_t^D, h_t^D \text{ observable })
        \end{split}
      \end{equation*} \par
      We set the starting time $t_0$ such that both $K_{t_0}^D$ and $h_{t_0}^D$ are observable. \par
      \noindent \textit{Step 3.}     We repeat the simulation and likelihood increment calculation for $m = 1, \dots , M$ and derive the simulated log likelihood increment for individual $i$ as follows:
      $$l_i = \log \left[ \sum_{m=1}^M \exp (l_i^m)\right]$$
  The total log likelihood is
  $$l = \sum_{i=1}^N l_i$$
      
  \end{document}
