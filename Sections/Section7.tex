% Conclusion section 7
% LaTeX path to the root directory of the current project, from the directory in which this file resides
% and path to econtexPaths which defines the rest of the paths like \FigDir
\providecommand{\econtexRoot}{}\renewcommand{\econtexRoot}{..}
\providecommand{\econtexPaths}{}\renewcommand{\econtexPaths}{\econtexRoot/TeXtools/econtexPaths}
% Mimicing professor's "ugly" solution to enable sharing base code between local LaTeX compilation and Overleaf

\providecommand{\FigDir}{\econtexRoot/FigDir}
\providecommand{\DataDir}{\econtexRoot/Data}
\providecommand{\SectionDir}{\econtexRoot/Sections}
\providecommand{\AppDir}{\econtexRoot/Appendices}
\providecommand{\TablesDir}{\econtexRoot/Tables}

\documentclass[\econtexRoot/ImaiKeane]{subfiles}
%\onlyinsubfile{\externaldocument{\econtexRoot/ImaiKeane}}

\begin{document}
\section{SUMMARY AND CONCLUSIONS}
\label{section:conclusions}
In this article, we use the NLSY79 data to estimate the intertemporal elasticity of substitution in labor supply in a framework where people explicitly take into account human capital accumulation when they make labor supply decisions. We explicitly solve the continuous variable dynamic programming problem for optimal consumption and labor supply decisions and use the derived emax function in a ML routine. Using the estimated parameters, we conduct simulation experiments to generate age-wage, and age-labor hours profiles. We also use the simulated data to estimate the i.e.s. parameter using the conventional OLS and the IV methods. \par
The results indicate that the ML method based on the full solution of the continuous stochastic dynamic programming problem gives an estimated elasticity of intertemporal substitution parameter of 3.820, which is comparable to the elasticity results discussed in the macroliterature. In contrast, in the microliterature, MaCurdy and Altonji have obtained IV estimates using the PSID that range from roughly 0.37 to 0.88 . Using the NLSY79, and applying the same IV procedure as MaCurdy and Altonji, we obtain elasticity estimates of 0.260 using the raw data and 0.142 using data with outliers removed. We also find that if we simulate data from the structural model, with the substitution elasticity set to 3.82, and use conventional methods (IV) to estimate this parameter, we obtain estimates that are severely biased towards zero. Thus, the main reason for our much higher estimate of the intertemporal elasticity of substitution when we use the full solution procedure is clearly our explicit inclusion of human capital accumulation in the model. \par
The simulated age profiles of wages and the marginal rate of substitution between labor supply and consumption imply that in the early stage of the agents' careers, the effective wage, which we define as the marginal rate of substitution, is as much as 2.0 times higher than the real wage, implying that at younger ages, even if observed wages are low, the high effective wage resulting from high returns to human capital accumulation induces agents to have high labor supply. However, as agents acquire experience and become older, the ratio of the marginal rate of substitution to the wage falls. Through this mechanism, the labor supply model with human capital accumulation is able to reconcile a high elasticity of substitution with the fact that wages have a pronounced hump shape over the life cycle whereas the hump in hours is much more modest. \par
 Finally, a word of caution is in order when one interprets the above results. Simulation results show that although the elasticity of intertemporal substitution was estimated to be around 3.8, the model does not imply that individuals change labor supply by a rate 3.8 times the rate of a wage change. On the contrary, the simulated hours response to a temporary wage increase of 2\% ranged from 0.6\% for young individuals to 4\% for individuals near retirement. One reason is that when young, human capital accumulation is an important factor in determining labor supply, so temporary wage changes have little effect on labor supply. Second, even temporary wage shocks in the model here have some persistence, as increased labor supply leads to higher wages in the future through the human capital production function. Hence, there is an income effect.
\end{document}
